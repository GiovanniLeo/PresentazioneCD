\documentclass{beamer}
\usetheme{Boadilla}
\usepackage{sansmathaccent}
\pdfmapfile{+sansmathaccent.map}
\usepackage{mathtools}
\usepackage{amsmath}
\usepackage{amsfonts}
\usepackage[utf8]{inputenc}

\title{Secure Compression and Pattern Matching Based on Burrows-Wheeler Transform}
\author{Raffaele Ceruso Giovanni Leo}

\date{\today}

\begin{document}
\begin{frame}
	\titlepage
\end{frame}
\begin{frame}
\frametitle{Protocollo}
Il protocollo prevede le seguenti fasi:

\textbf{$(1)$ Inizializzazione.} Il client genera le strutture dati \textit{c} e \textit{occ}. \textit{occ} è generata nel seguente modo.

\begin{itemize}
	\item \textit{superblock}:
	
	scegli un primo $ p_{1} $ leggermente più grande
	di n = L2, un numero diverso da zero $ \beta_{1} $ e scegli un generatore
	$g_{1}$ di $ Z^{*}_{p_{1}} $. Il client mantiene $ g_{1} \beta_{1} $ segreti. Per \textit{superblock} $ i $, memorizziamo quei dati nel $ (g^{i}_{1}\beta_{1})^{th} mod$ $p_{1}$
	(omettiamo $mod$ $ p_{1}$ nella parte restante) della tabella.
	Per le entries vuote inseriamo alcuni numeri casuali.
\end{itemize}
\end{frame}
\begin{frame}
\frametitle{Protocollo}
Il protocollo prevede le seguenti fasi:

\textbf{$(1)$ Inizializzazione.} Il client genera le strutture dati \textit{c} e \textit{occ}. \textit{occ} è generata nel seguente modo.

\begin{itemize}
	\item \textit{block}:
	
	scegli un primo $ p_{2} $ che è poco più grande di $n/L^{2}$,
	un numero diverso da zero $ \beta_{2} $ e scegli un generatore $g_{2}$ di $ Z^{*}_{p_{2}} $.
	Il client mantiene $ g_{2} \beta_{2} $ segreti. Per \textit{block} $ i $, memorizziamo quei dati nel $ (g^{i}_{2}\beta_{2})^{th} mod$ $p_{2}$
	(omettiamo $mod$ $ p_{2}$ nella parte restante) della tabella.
	Per le entries vuote inseriamo alcuni numeri casuali.
\end{itemize}
\end{frame}
\begin{frame}
\frametitle{Protocollo}
Il protocollo prevede le seguenti fasi:

\textbf{$(1)$ Inizializzazione.} Il client genera le strutture dati \textit{c} e \textit{occ}. \textit{occ} è generata nel seguente modo.

\begin{itemize}
	\item \textit{Hash\_value}:
	
	Invece di memorizzare immagini di \textit{MTF}
	tabelle, memorizziamo il valore hash di $ (i-1)^{th} $ block
	per il blocco $ i^{th} $. Il valore hash di $ (i-1)^{th} $ block
	è memorizzato nella $ (g^{i}_{2}\beta_{2})^{th}$ entry della tabella $Hash_value$ .
	Per le entries vuote inseriamo alcuni numeri casuali.
\end{itemize}
\end{frame}
\begin{frame}
\frametitle{Protocollo}
Il protocollo prevede le seguenti fasi:

\textbf{$(1)$ Inizializzazione.} Il client genera le strutture dati \textit{c} e \textit{occ}. \textit{occ} è generata nel seguente modo.

\begin{itemize}
	\item \textit{Block\_inner}:
	

	La riga \textit{Block\_inner}($mtf[i],BZ_{i}, e, h-L*i$) viene spostato su \textit{Block\_inner}\textit{(Hash\_value}[$g^{i}_{2}\beta_{2}], BZ_{g^{i}{\beta_{2}}}, e, \tau(key, h-L*i))$
	dove $i = \lfloor h/L\rfloor$ e $\tau$ : $Z_{L} $ $\rightarrow$ $Z_{L} $ è una funzione di permutazione casuale.
\end{itemize}
\end{frame}
\begin{frame}
\frametitle{Protocollo}
Il protocollo prevede le seguenti fasi:

\textbf{$(1)$ Inizializzazione.}
	Pertanto, per calcolare $occ(e, h)$, il client deve inviare un vettore di posizione $pv(h) = (g_{1}^{h/L^{2}} \beta_{1}, g_{2}^{h/L} \beta_{2}, \tau(key, h-[h/L] * L) )$ al server.\\ 
	
	Pertanto l'algoritmo è Algo = sBWT + bMTF + RLE + PC +P dove P è la permutazione. Dopo aver criptato i dati in $occ$ con AHE il client li invia al server il quale mantiene $c$. Successivamente sceglie un numero adeguato R come round di comunicazione.
\end{frame}

\begin{frame}
\frametitle{Protocollo}
Il protocollo prevede le seguenti fasi:\\
\textbf{$(2)$ Client:} invia un vettore posizione $pos$ e legge il vettore $V$ in accordo di $P[i-1]$
\end{frame}

\begin{frame}
\frametitle{Protocollo}
Il protocollo prevede le seguenti fasi:\\
\textbf{$(3)$ Server:} per ogni carattere dell'alfabeto e del $y^{th}$ vettore in $pos$, computa $B_{y}[\zeta]$ e ritorna ${a_{y}}_{y\in[1,4]}$
\end{frame}

\begin{frame}
\frametitle{Protocollo}
Il protocollo prevede le seguenti fasi:\\
\textbf{$(4)$ Client:} trova l'aspettato $a_{i1}$, $a_{i2}$ $\in {a_{y}}_{y\in[1,4]} $ e computa
\begin{itemize}
	\item $occ(P[i-1], begin-1)$
	\item $occ(P[i-1], end)$
	\item $begin$
	\item $end$
\\	Se $begin \leq end$ e $i \geq 2 $ vai allo step 2, altrimenti invia $pos$ al server. Se il round di comunicazione è $R$ vai al prossimo step.
\end{itemize}
\end{frame}

\begin{frame}
\frametitle{Protocollo}
\textbf{$(5)$ Client:} se $end < begin$ il pattern non è stato trovato, altrimenti l'occorrenza del pattern è $end - begin +1$
\end{frame}

\end{document}